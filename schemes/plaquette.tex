\documentclass{standalone}
\thispagestyle{empty}
\usepackage{tikz}
\usetikzlibrary{decorations.markings}

% definition of all the colors
\usepackage{xcolor}
%New colors defined below
\definecolor{codecomment}{rgb}{0,0.6,0} % green
\definecolor{codegray}{rgb}{0.5,0.5,0.5} % gray
\definecolor{codestring}{rgb}{0.4,0.84,0.93} % blue
\definecolor{codebackgroundcolour}{rgb}{0.95,0.95,0.92}
\definecolor{corange}{RGB}{255, 70, 0} % orange
\definecolor{cyellow}{RGB}{209, 153, 0} % yellow
\definecolor{cblue}{RGB}{64, 128, 255} % blue
\definecolor{cbrown}{RGB}{153, 102, 51} % brown
\definecolor{cpink}{RGB}{255, 0, 255} % pink
\definecolor{cred}{RGB}{255, 64, 0} % red
\definecolor{cgreen}{RGB}{0, 191, 0} % green
\definecolor{clightblue}{RGB}{191, 217, 255} % light blue
\definecolor{clightgreen}{RGB}{224, 255, 224} % light green
\definecolor{clightpink}{RGB}{255, 230, 255} % light pink
\definecolor{cdarkblue}{RGB}{0, 0, 255} % dark blue

\begin{document}
\begin{tikzpicture}

\def\step{2}
\node[circle,inner sep=0pt,minimum size=0pt,label=below left :{$x$}] (A) at ( \step*0, \step*0 ) {};
\node[circle,inner sep=0pt,minimum size=0pt,label=below right:{$x + n_2 \epsilon$}] (B) at ( \step*1, \step*0 ) {};
\node[circle,inner sep=0pt,minimum size=0pt,label=above right:{$x + n_2 \epsilon + n_1 \epsilon$}] (C) at ( \step*1, \step*1 ) {};
\node[circle,inner sep=0pt,minimum size=0pt,label=above left :{$x + n_1 \epsilon$}] (D) at ( \step*0, \step*1 ) {};

\begin{scope}[thick,decoration={
    markings,
    mark=at position 0.5 with {\arrow{>}}}
    ] 
    \draw[postaction={decorate}] (A)--(B);
    \draw[postaction={decorate}] (B)--(C);
    \draw[postaction={decorate}] (C)--(D);
    \draw[postaction={decorate}] (D)--(A);
\end{scope}

\draw[->] (-4,0) -- (-4,1) node[midway,left]  {$n_1$};
\draw[->] (-4,0) -- (-3,0) node[midway,below] {$n_2$};

\node[circle,inner sep=0pt,minimum size=0pt] (E) at ( 6,0 ) {};

\end{tikzpicture}
\end{document}