\documentclass{standalone}
\thispagestyle{empty}
\usepackage{tikz}
\usetikzlibrary{decorations.markings}

% definition of all the colors
\usepackage{xcolor}
%
% genertated by ./compile.py
%

\definecolor{codegreen}{RGB}{0, 153, 0}
\definecolor{codegray}{RGB}{127, 127, 127}
\definecolor{codeblue}{RGB}{102, 214, 237}
\definecolor{codekeyword}{RGB}{249, 36, 114}
\definecolor{codecomment}{RGB}{127, 127, 127}
\definecolor{backcolor}{RGB}{242, 242, 235}
\definecolor{linkcolor}{RGB}{102, 0, 0}
\definecolor{corange}{RGB}{255, 70, 0}
\definecolor{cyellow}{RGB}{209, 153, 0}
\definecolor{cblue}{RGB}{64, 128, 255}
\definecolor{cbrown}{RGB}{153, 102, 51}
\definecolor{cpink}{RGB}{255, 0, 255}
\definecolor{cred}{RGB}{255, 64, 0}
\definecolor{cgreen}{RGB}{0, 191, 0}
\definecolor{clightblue}{RGB}{191, 217, 255}
\definecolor{cturquois}{RGB}{0, 255, 255}
\definecolor{cpurple}{RGB}{128, 0, 255}
\definecolor{clightgreen}{RGB}{175, 255, 175}
\definecolor{clightpink}{RGB}{255, 175, 255}
\definecolor{cdarkblue}{RGB}{0, 0, 255}
\definecolor{cdarkred}{RGB}{255, 0, 0}
\definecolor{cdarkgreen}{RGB}{0, 255, 0}


\begin{document}
\begin{tikzpicture}

\def\step{2}
\node[circle,inner sep=0pt,minimum size=0pt,label=below left :{$x$}] (A) at ( \step*0, \step*0 ) {};
\node[circle,inner sep=0pt,minimum size=0pt,label=below right:{$x + n_2 \epsilon$}] (B) at ( \step*1, \step*0 ) {};
\node[circle,inner sep=0pt,minimum size=0pt,label=above right:{$x + n_2 \epsilon + n_1 \epsilon$}] (C) at ( \step*1, \step*1 ) {};
\node[circle,inner sep=0pt,minimum size=0pt,label=above left :{$x + n_1 \epsilon$}] (D) at ( \step*0, \step*1 ) {};

\begin{scope}[thick,decoration={
    markings,
    mark=at position 0.5 with {\arrow{>}}}
    ] 
    \draw[postaction={decorate}] (A)--(B);
    \draw[postaction={decorate}] (B)--(C);
    \draw[postaction={decorate}] (C)--(D);
    \draw[postaction={decorate}] (D)--(A);
\end{scope}

\draw[->] (-4,0) -- (-4,1) node[midway,left]  {$n_1$};
\draw[->] (-4,0) -- (-3,0) node[midway,below] {$n_2$};

\node[circle,inner sep=0pt,minimum size=0pt] (E) at ( 6,0 ) {};

\end{tikzpicture}
\end{document}